

\documentclass[final,5p,times,twocolumn,authoryear]{elsarticle}


\usepackage{amssymb}
\usepackage{lipsum}


\journal{Astronomy $\&$ Computing}


\begin{document}

\begin{frontmatter}



\title{College of Engineering Pune}

      

\author[first]{From 1854}
\affiliation[first]{organization={Excellence is an art won by training and habituation},%Department and Organization
            addressline={}, 
            city={Earth},
            postcode={}, 
            state={},
            country={}}

\begin{abstract}

Example abstract for astronomy and computing journal. Here you provide a brief summary of the research and the results. 
\end{abstract}


\begin{keyword}

keyword \sep keyword \sep keyword \sep keyword
or \MSC[2008] code \sep code (2000 is the default)

\end{keyword}


\end{frontmatter}
\tableofcontents

%% \linenumbers

%% main text
\section{Introduction}
\label{introduction}

Here is where you provide an introduction to work and some background. For example building on previous work of image enhancment in optical astronomy \citep{vojtekova2021learning}, \cite{sweere2022deep} developed a method to improve the resolution of X-ray images from XMM-Newton to provide similar spatial resolution to Chandra.

\section{Campus}
\lipsum[1-4]
\section{Acadamics}
\lipsum[1-4]

\section{Hostels}
\lipsum[1-4]

\section{Placements}
\lipsum[1-4]


\section*{Acknowledgements}
Thanks to ...

\appendix

\section{Admissions}


\section{Cut off}



\bibliographystyle{elsarticle-harv} 
\bibliography{example}

\end{document}

\endinput
%%

