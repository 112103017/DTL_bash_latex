\documentclass[12pt]{article}
\usepackage{amsmath}
\author{\large Hariom Badarkhe}
\date{\today}
\begin{document}
\raggedright
\title{\huge \bf DTL Assignment-1}
\maketitle
\newpage
\pagenumbering{gobble}
\section{\Huge Sections And Subsections}
\vspace{5mm}
\subsection{\large Universe}
The universe is everything. It includes all of space, and all the matter and energy that space contains. It even includes time itself and, of course, it includes you.
Earth and the Moon are part of the universe, as are the other planets and their many dozens of moons. Along with asteroids and comets, the planets orbit the Sun. The Sun is one among hundreds of billions of stars in the Milky Way galaxy, and most of those stars have their own planets, known as exoplanets.

The Milky Way is but one of billions of galaxies in the observable universe — all of them, including our own, are thought to have supermassive black holes at their centers. All the stars in all the galaxies and all the other stuff that astronomers can’t even observe are all part of the universe. It is, simply, everything.
\subsubsection{ How old is the Universe }
Though the universe may seem a strange place, it is not a distant one. Wherever you are right now, outer space is only 62 miles (100 kilometers) away. Day or night, whether you’re indoors or outdoors, asleep, eating lunch or dozing off in class, outer space is just a few dozen miles above your head. It’s below you too. About 8,000 miles (12,800 kilometers) below your feet — on the opposite side of Earth — lurks the unforgiving vacuum and radiation of outer space.

In fact, you’re technically in space right now. Humans say “out in space” as if it’s there and we’re here, as if Earth is separate from the rest of the universe. But Earth is a planet, and it’s in space and part of the universe just like the other planets. It just so happens that things live here and the environment near the surface of this particular planet is hospitable for life as we know it. Earth is a tiny, fragile exception in the cosmos. For humans and the other things living on our planet, practically the entire cosmos is a hostile and merciless environment.
\subparagraph{ What is the universe made of?}
The universe contains all the energy and matter there is. Much of the observable matter in the universe takes the form of individual atoms of hydrogen, which is the simplest atomic element, made of only a proton and an electron (if the atom also contains a neutron, it is instead called deuterium). Two or more atoms sharing electrons is a molecule. Many trillions of atoms together is a dust particle. Smoosh a few tons of carbon, silica, oxygen, ice, and some metals together, and you have an asteroid. Or collect 333,000 Earth masses of hydrogen and helium together, and you have a Sun-like star.


\subsubsection{The Interpreter}
Clearly the interpreter must perform the operations that are specific to the language it is interpreting, L . However, even given the diversity of languages, it is possible to discern types of operation and an \textit{“execution method”} common to all interpreters.
\subsubsection{An Example of an Abstract Machine: The Hardware Machine}
From what has been said so far, it should be clear that the concept of abstract machine can be used to describe a variety of different systems, ranging from physical machines right up to the World Wide Web.\\
As a first example of an abstract machine, let us consider the concrete case of a
conventional physical machine such as that in Fig. 1.3. It is physically implemented using logic circuits and electronic components. Let us call such a machine MHL H and let L H be its machine language.



\end{document}